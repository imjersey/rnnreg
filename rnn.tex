
% Use the following line  only  if you're still using LaTeX 2.09.
%\documentstyle[icml2014,epsf,natbib]{article}
% If you rely on Latex2e packages, like most moden people use this:
\documentclass{article}
% use Times
\usepackage{times}
 % For figures
\usepackage{graphicx} % more modern
%\usepackage{epsfig} % less modern
\usepackage{subfigure} 
\usepackage{footnote}
\usepackage{amsfonts}

% For citations
\usepackage{natbib}
\usepackage{comment}
\usepackage{multirow}


% For algorithms
\usepackage{algorithm}
\usepackage{algorithmic}
\usepackage{amsmath}

\setlength{\abovedisplayskip}{0cm}
\setlength{\belowdisplayskip}{0cm}

\usepackage[compact]{titlesec}
\titlespacing{\section}{0pt}{0.5ex}{0.3ex}
\titlespacing{\subsection}{0pt}{0.2ex}{0ex}
\titlespacing{\subsubsection}{0pt}{0.1ex}{0ex}

\newcommand{\startcompact}[1]{\par\vspace{-0.75em}\begin{#1}%
  \allowdisplaybreaks\ignorespaces}

\newcommand{\stopcompact}[1]{\end{#1}\ignorespaces}

\usepackage{paralist}

\makeatletter
\ifcase \@ptsize \relax% 10pt
  \newcommand{\miniscule}{\@setfontsize\miniscule{4}{5}}% \tiny: 5/6
\or% 11pt
  \newcommand{\miniscule}{\@setfontsize\miniscule{5}{6}}% \tiny: 6/7
\or% 12pt
  \newcommand{\miniscule}{\@setfontsize\miniscule{5}{6}}% \tiny: 6/7
\fi
\makeatother

\newcommand {\aplt} {\ {\raise-.5ex\hbox{$\buildrel<\over\sim$}}\ }

\newcommand{\eqn}[1]{Eqn.~\ref{eqn:#1}}
\newcommand{\fig}[1]{Fig.~\ref{fig:#1}}
\newcommand{\tab}[1]{Table~\ref{tab:#1}}
\newcommand{\secc}[1]{Section~\ref{sec:#1}}
\def\etal{{\textit{et~al.~}}}
\newcommand{\BigO}[1]{\ensuremath{\operatorname{O}\left(#1\right)}}
\usepackage[symbol*]{footmisc}

\DefineFNsymbolsTM{myfnsymbols}{% def. from footmisc.sty "bringhurst" symbols
  \textasteriskcentered *
  \textdagger    \dagger
  \textdaggerdbl \ddagger
  \textsection   \mathsection
  \textbardbl    \|%
  \textparagraph \mathparagraph
}%


% As of 2011, we use the hyperref package to produce hyperlinks in the
% resulting PDF.  If this breaks your system, please commend out the
% following usepackage line and replace \usepackage{icml2014} with
% \usepackage[nohyperref]{icml2014} above.
\usepackage{hyperref}

% Packages hyperref and algorithmic misbehave sometimes.  We can fix
% this with the following command.
\newcommand{\theHalgorithm}{\arabic{algorithm}}

% Employ the following version of the ``usepackage'' statement for
% submitting the draft version of the paper for review.  This will set
% the note in the first column to ``Under review.  Do not distribute.''
%\usepackage{icml2014} 
% Employ this version of the ``usepackage'' statement after the paper has
% been accepted, when creating the final version.  This will set the
% note in the first column to ``Proceedings of the...''
\usepackage[accepted]{icml2014}

\begin{document} 

\twocolumn[
\icmltitle{Recurrent neural network regularization}

\icmlauthor{Wojciech Zaremba}{woj.zaremba@gmail.com}
\vskip -0.03in
\icmladdress{Google \& New York University}
\vskip -0.08in
\icmlauthor{Ilya Sutskever}{ilyasu@google.com}
\vskip -0.03in
\icmladdress{Google}
\vskip -0.08in
\icmlauthor{Oriol Vinyals}{vinyals@google.com}
\vskip -0.03in
\icmladdress{Google}


\icmlkeywords{natual language processing, recurrent neural networks, language model, LSTMs, speech recognition, machine translation}

\vskip 0.3in
]

\begin{abstract} 
  We present a simple regularization technique for Recurrent Neural
  Networks (RNNs) with Long Short-Term Memory (LSTM) units.  The
  technique is based on dropout and gives a tremendous reduction in
  overfitting.  We show that it is useful in a variety of sequence
  modeling problems that include language modeling, speech recognition, and
  machine translation.
\end{abstract} 

\section{Introduction}

RNNs yield the state of the art performance on many sequence modeling
tasks, including language modeling \cite{mikolov2012statistical}, speech recognition
\cite{graves2013speech}, and machine translation \cite{cho2014learning}.
However, there have been no good ways of regularizing them. As a
result, practical applications often use RNNs that are too small, as
large RNNs tend to overfit.  To date, existing regularization methods
give relatively small improvements on RNNs
\cite{graves2013generating}.  Dropout is a highly effective way of
regularizing feedforward neural networks
\cite{srivastava2013improving} that had enjoyed considerable
success. However, it is not clear how to use dropout on RNNs,
because a naive application of dropout does not yield good results.
In this work, we show how to correctly apply dropout to LSTMs, and
demonstrate that this results in great reduction in overfitting.

\section{Related work}

Dropout \cite{srivastava2013improving} is a recently regularization
method that has enjoyed a lot of success in applications of
feed-forward neural networks.  While there has been a lot of work
extending Dropout \cite{wang2013fast, wan2013regularization}, there
has been relatively little research so far in applying dropout to
RNNs. The only paper on the topic is \cite{bayer2013fast} which
focuses on ``marginalized dropout'' (from \cite{wang2013fast}) which
is a noiseless deterministic approximation to conventional dropout.
\cite{bayer2013fast} claims that conventional dropout cannot be
successfully used with RNNs because its interaction with recurrence
causes large variance which hurts learning and causes poor
convergence. In this work, we show that a specific application of
dropout greatly reduces the overfitting of RNNs.

There has been extensive work on using RNNs for language modeling
\cite{mikolov2012statistical, sutskever2013training}. Moreover, there
have been a number of architectural variations on the RNN that are
better suited for learning on data with long term dependencies
\cite{hochreiter1997long, graves2009novel, cho2014learning,
  jaeger2007optimization, koutnik2014clockwork}.  This work focuses on the LSTM which
is the most widespread variant of the RNN, although it is likely that 
our findings are valid for other models.

In this paper, we focus on the following RNN tasks: language
modeling, speech recognition, and machine translation.  Language
modeling is the first task where RNNs achieved substantial success
\cite{mikolov2010recurrent, mikolov2011strategies,
  pascanu2013construct}.  RNNs have also been successfully used for
speech recognition \cite{robinson1996use, graves2013speech} and have
been applied to machine translation, where they are typically
used for language modeling, re-ranking, or phrase modeling
\cite{SVL2014,cho2014learning,chow1987byblos,mikolov2013exploiting}.

\section{Regularizing RNN with LSTM cells}

In this section we describe the deep LSTM \ref{sec:lstm}. Next, 
we show how to regularize them \ref{sec:reg}, and we provide an intuition
for why our regularization scheme works.

We use lowerscript to denote timesteps and upperscript to denote 
layer number.  All our states are $n$-dimensional.  Let $h^l_k
\in \mathbb{R}^{n}$ be a hidden state in layer $l$ in step
$k$. Moreover, let $T_{n,m}:\mathbb{R}^{n} \rightarrow \mathbb{R}^{m}$
be a linear transform and with a bias ($Wx + b$ for some $W$ and $b$).
Let $\odot$ be a element-wise multiplication and let $h^0_k$ be an
input word vector.  We use the activations $h^{L}_k$ to predict $y_k$,
since $L$ is the number of layers in our deep LSTM.

\subsection{Long-short term memory units}
\label{sec:lstm}

The RNNs dynamics can be described in terms of deterministic transitions
from previous hidden states to the hidden states in the next step. 
The transition is a function
\begin{align*}
  &\text{RNN} : h^{l-1}_k, h^l_{k-1} \rightarrow h^l_k
\end{align*}

For classical RNNs, this function is given by
\begin{align*}
  h^l_k = f(T_{n,n}h^{l-1}_k + T_{n,n}h^l_{k-1}) \text{, where $f \in \{\mathrm{sigm}, \tanh\}$ }
\end{align*}

The LSTM has relatively complicated dynamics that make it easy to
``memorize'' information for extended number of time steps.  The
``long term'' memory is stored in a vector of \emph{memory cells}
$c^l_k \in \mathbb{R}^n$.  Although there are many LSTM architectures
that differ in their connectivity sturcutre and activation functions,
all LSTM architectures have explicit memory cells that make it easy to
store information for extended periods of time.  The LSTM can decide
to overwrite this information, retrieve it, or keep it for the next time
step.  The LSTM architecture used in our experiments is given by the
following equations \cite{graves2013speech}:
\begin{align*}
&\text{LSTM} : h^{l-1}_k, h^l_{k-1}, c^l_{k - 1} \rightarrow h^l_k, c^l_k\\
&\begin{pmatrix}i\\f\\o\\g\end{pmatrix} =
  \begin{pmatrix}\mathrm{sigm}\\\mathrm{sigm}\\\mathrm{sigm}\\\tanh\end{pmatrix}
  T_{2n,4n}\begin{pmatrix}h^{l - 1}_k\\h^l_{k-1}\end{pmatrix}\\
&c^l_k = f \odot c^l_{k-1} + i \odot g\\
&h^l_k = o \odot \tanh(c^l_k)\\
\end{align*}
In these equations, $\mathrm{sigm}$ and $\tanh$ are applied
element-wise. Diagram \ref{fig:lstm} illustrates the LSTM
equations.


\begin{figure}
  \begin{center}
    \begin{picture}(200, 130)
      \put(0, 0){\framebox(180, 100){}}
      \put(90, 50){\circle{16}}
      \put(86.5, 48){$\mathbf c_t$}
      \put(84, 60){{\scriptsize Cell}}

      \put(90, 32){\circle{6.5}}
      \put(87.25, 30.5){{\tiny $\times$}}

      \put(90, 12){\circle{16}}
      \put(86.5, 10){{\small $f$}}
      \put(100, 10){{\scriptsize Forget gate}}

      \put(90, 20){\vector(0, 0){8}}

      \put(85, 44){\vector(2, 3){0}}
      \qbezier(86, 32)(80, 36.5)(83, 41)

      \qbezier(96, 34)(100, 36.5)(95.5, 43.5)
      \put(93.5, 31.5){\vector(-1, -1){0}}
      
      \put(82, -8){\vector(1, 2){6}}
      \put(72.5, -13){{\small $h_{k-1}^{l}$}}
      \put(98, -8){\vector(-1, 2){6}}
      \put(99, -13){{\small $h_{k}^{l-1}$}}

      \put(30, 87){\circle{16}}
      \put(28, 85){{\small $i$}}
      \put(6, 85){{\scriptsize $\begin{matrix}\text{Input}\\\text{gate}\end{matrix}$}}

      \put(21.5, 105){\vector(1, -2){5}}
      \put(12.5, 108){{\small $h_{k-1}^{l}$}}
      \put(39.5, 107){\vector(-1, -2){6}}
      \put(37, 108){{\small $h_{k}^{l-1}$}}

      \put(147, 87){\circle{16}}
      \put(144.5, 85){{\small $o$}}
      \put(156, 85){{\scriptsize $\begin{matrix}\text{Output}\\\text{gate}\end{matrix}$}}
        
      \put(138.5, 105){\vector(1, -2){5}}
      \put(129.5, 108){{\small $h_{k-1}^{l}$}}
      \put(156.5, 107){\vector(-1, -2){6}}
      \put(154, 108){{\small $h_{k}^{l-1}$}}

      \put(17, 50){\circle{16}}
      \put(15, 48){{\small $g$}}
      \put(1, 28){{\scriptsize $\begin{matrix}\text{Input}\\\text{modulation}\\\text{gate}\end{matrix}$}}

      \put(53.5, 50){\circle{6.5}}
      \put(50.75, 48.5){{\tiny $\times$}}

      \put(57, 50){\vector(1, 0){25}}
      \put(25, 50){\vector(1, 0){25}}
      \put(35, 80){\vector(2, -3){17.5}}

      \put(147, 50){\circle{6.5}}
      \put(144.25, 48.5){{\tiny $\times$}}
      \put(98, 50){\vector(1, 0){45.25}}
      \put(150.5, 50){\vector(1, 0){38}}

      \put(147, 79){\vector(0, -1){25.5}}
      \put(190, 47){${\mathbf h^l_k}$}


      \put(-20, 40){{\small $h_{k-1}^{l}$}}
      \put(-20, 56){{\small $h_{k}^{l-1}$}}
      \put(-10, 44){\vector(4, 1){19}}
      \put(-10, 58){\vector(4, -1){19}}


    \end{picture}
  \end{center}
  \caption{Graphical representation of LSTM memory cells used in this paper (there are minor differences in compare to \cite{graves2013generating}.}
  \label{fig:lstm}
\end{figure}


\subsection{Regularization with Dropout} 
\label{sec:reg}

The main contribution of this paper is the discovery that carefully
placed dropout tremendously improves the generalization ability of LSTMs.
To be effective, dropout must be placed on the non-recurrent connections
\ref{fig:reg}.  The following equation describes it more precisely,
where ${\bf D}$ is the dropout operator that sets a random subset of its 
argument to zero:

\begin{align*}
&\begin{pmatrix}i\\f\\o\\g\end{pmatrix} =
  \begin{pmatrix}\mathrm{sigm}\\\mathrm{sigm}\\\mathrm{sigm}\\\tanh\end{pmatrix}
  T_{2n,4n}\begin{pmatrix}{\bf D}(h^{l - 1}_k)\\h^l_{k-1}\end{pmatrix}\\
&c^l_k = f \odot c^l_{k-1} + i \odot g\\
&h^l_k = o \odot \tanh(c^l_k)\\
\end{align*}


\begin{figure}
  \begin{center}
    \begin{picture}(150, 200)
      \multiput(0,0)(35, 0){6}{
        \put(-25, 45){\vector(1, 0){25}}
        \put(-25, 100){\vector(1, 0){25}}
      }
      \multiput(0,0)(35, 0){5}{
        \put(0, 0){
          \put(0, 85){\framebox(10, 30){}}
          \put(0, 30){\framebox(10, 30){}}
          \multiput(0,0)(0, 5){4}{
            \put(5, 7){\line(0, 0){2}}
            \put(5, 60){\line(0, 0){2}}
            \put(5, 115){\line(0, 0){2}}
          }
          \put(5, 30){\vector(0, 0){0.1}}
          \put(5, 85){\vector(0, 0){0.1}}
          \put(5, 138){\vector(0, 0){0.1}}
        }
      }
      \put(-2, 0){\makebox{$x_{i-2}$}}
      \put(33, 0){\makebox{$x_{i-1}$}}
      \put(71, 0){\makebox{$x_{i}$}}
      \put(103, 0){\makebox{$x_{i+1}$}}
      \put(138, 0){\makebox{$x_{i+2}$}}
      \put(-2, 142){\makebox{$y_{i-2}$}}
      \put(33, 142){\makebox{$y_{i-1}$}}
      \put(71, 142){\makebox{$y_{i}$}}
      \put(103, 142){\makebox{$y_{i+1}$}}
      \put(138, 142){\makebox{$y_{i+2}$}}
    \end{picture}
  \end{center}
  \caption{Regularized multilayer RNN. Dashed arrows indicate connections with applied dropout, while
  solid lines indicate connections where dropout is not applied.}
  \label{fig:reg}
\end{figure}

The dropout operator corrupts the information carried by the units,
which forces them to perform their intermediate computations in a more
robust manner. At the same time, we do not want to erase all of the
information conveyed by the units. We would especially like the units
to remember events that occurred many timeseps in the past. Figure
\ref{fig:flow} shows a possible flow of information from an event that
occurred at $x_{i-2}$ to the prediction in the step $i+2$. We can see
that the information is corrupted by dropout only $L + 1$ times, and
it is independent of how far in past event occurred.  A naive
application of dropout would perturb the recurrent connections or the
recurrent hidden state, which would greatly reduce the LSTM's memory
capacity.  By not using dropout on the recurrent connections, the LSTM
is able to get most of the benefit of dropout without sacrifising its
valuable ability to learn and store information for long periods of
time.


\begin{figure}
  \begin{center}
    \begin{picture}(150, 200)
      \multiput(0,0)(35, 0){6}{
        \put(-25, 45){\vector(1, 0){25}}
        \put(-25, 100){\vector(1, 0){25}}
      }
      \multiput(0,0)(35, 0){5}{
        \put(0, 0){
          \put(0, 85){\framebox(10, 30){}}
          \put(0, 30){\framebox(10, 30){}}
          \multiput(0,0)(0, 5){4}{
            \put(5, 7){\line(0, 0){2}}
            \put(5, 60){\line(0, 0){2}}
            \put(5, 115){\line(0, 0){2}}
          }
          \put(5, 30){\vector(0, 0){0.1}}
          \put(5, 85){\vector(0, 0){0.1}}
          \put(5, 138){\vector(0, 0){0.1}}
        }
      }
      \put(-2, 0){\makebox{$x_{i-2}$}}
      \put(33, 0){\makebox{$x_{i-1}$}}
      \put(71, 0){\makebox{$x_{i}$}}
      \put(103, 0){\makebox{$x_{i+1}$}}
      \put(138, 0){\makebox{$x_{i+2}$}}
      \put(-2, 142){\makebox{$y_{i-2}$}}
      \put(33, 142){\makebox{$y_{i-1}$}}
      \put(71, 142){\makebox{$y_{i}$}}
      \put(103, 142){\makebox{$y_{i+1}$}}
      \put(138, 142){\makebox{$y_{i+2}$}}

       
      {\linethickness{0.6mm}
        \put(5, 7){\line(0, 0){38}}
        \put(4, 45){\line(1, 0){105}}
        \put(110, 44){\line(0, 0){56}}
        \put(109, 100){\line(1, 0){35}}
        \put(145, 99){\line(0, 0){35}}
      }
    \end{picture}
  \end{center}
  \caption{Thick line indicates an exemplary information flow in RNN. Information flow line is crossed $L + 1$ times, where $L$ is depth of network.}
  \label{fig:flow}
\end{figure}


\section{Experiments}

We present here results in there domains: language modeling \ref{sec:lang}, 
speech recognition \ref{sec:speech}, and machine translation \ref{sec:trans}.

\subsection{Language modeling}
\label{sec:lang}

We have conducted word-level prediction experiments on Penn tree bank
(PTB) dataset \cite{marcus1993building}.  This dataset consists of $929$k
training words, $73$k validation words, and $82$k test words. It has
$10$k words in vocabulary. We have trained two various size regularized LSTMs model.
Both models are a two-layer LSTM unrolled for $35$ steps. We set mini-batch to 20.

\begin{table}[t]
  \small
  \centering
  \renewcommand{\arraystretch}{1.15}
  \begin{tabular}{lll}
    \hline
     Model & Validation set & Test set \\
    \hline
    \multicolumn{3}{c}{A single model} \\
    \hline
    \cite{pascanu2013construct} & & 107.5 \\
    \cite{chenglanguage} & & 100.0 \\
    Non-regularized LSTM & 120.7 & 114.5 \\
    Medium Regularized LSTM & 86.2 & 82.7 \\
    Large Regularized LSTM & 82.2 & {\bf 78.4} \\
    \hline
    \multicolumn{3}{c}{Model averaging} \\
    \hline
    \cite{mikolov2012statistical} & 83.5 & 89.4 \\
    \cite{chenglanguage} & & 80.6 \\
    2 non-regularized LSTMs & 100.4 & 96.1 \\
    5 non-regularized LSTMs & 87.9 & 84.1 \\
    10 non-regularized LSTMs & 83.5 & 80.0 \\
    2 medium regularized LSTMs & 80.6 & 77.0 \\
    5 medium regularized LSTMs & 76.7 & 73.3 \\
    10 medium regularized LSTMs & 75.2 & 72.0 \\
    2 large regularized LSTMs & 76.9 & 73.6 \\
    10 large regularized LSTMs & 72.8 & 69.5 \\
    38 large regularized LSTMs & 71.9 & {\bf 68.7} \\
    \hline
    \multicolumn{3}{c}{Model averaging with dynamic RNNs} \\
    \hline
    \cite{mikolov2012context} & & 72.9 \\
    \hline
  \end{tabular}
  \caption{Word-level perplexity on Penn-tree-bank dataset.}
  \label{tab:ptb}
\end{table}

\begin{figure}
\line(1,0){235}

  {\footnotesize
  \textit{the meaning of life is} that only if an end would be of the whole supplier. widespread rules are regarded as the companies of refuses to deliver. in balance of the nation 's information and loan growth associated with the carrier thrifts are in the process of slowing the seed and commercial paper.}
\line(1,0){235}

  {\footnotesize
\textit{the meaning of life is} nearly in the first several months before the government was addressing such a move as president and chief executive of the nation past from a national commitment to curb grounds. meanwhile the government invests overcapacity that criticism and in the outer reversal of small-town america.}

\line(1,0){235}
  \caption{Some interesting samples drawn from large regularized model conditioned on ``The meaning of life is''. We have removed ``unk'', ``N'', ``\$'' from possible outcomes.}
  \label{fig:meaning}
\end{figure}



Medium size model has 650 units per layer
whose parameters are initialized uniformly in $[-0.05,
  0.05]$. We apply $50\%$ dropout on the non-recurrent connections. We
train for $39$ epochs, starting with learning rate of $1$, and after
$6$ epochs we decrease it by a factor of $1.2$ in every epoch. We
clip the norm of the gradients
(normalized by minibatch size) at $5$. 


Large size model has 1500 units per layer
whose parameters are initialized uniformly in $[-0.04,
  0.04]$. We apply $65\%$ dropout on the non-recurrent connections. We
train for $55$ epochs, starting with learning rate of $1$, and after
$14$ epochs we decrease it by a factor of $1.15$ in every epoch. We
clip the norm of the gradients
(normalized by minibatch size) at $10$. 


Table \ref{tab:ptb} compares various models with our models, and figure \ref{fig:meaning}
shows samples draw from a single large size regularized model.


\subsection{Speech recognition}
\label{sec:speech}

Deep Neural Networks have been used for acoustic modeling for more
than half a century (the reader is encouraged to read
\cite{BourlardASR} for a good review). Acoustic modeling is a key
component in mapping acoustic signals to sequences of words, as it
models $p(s_t|X)$ where $s_t$ is the phonetic state at time t, and $X$
the acoustic observation. Recent work has shown that LSTMs are very
powerful models for acoustic modeling \cite{sak2014speech}, and much
smaller networks (in number of parameters) are able to overfit the
data more easily. The primary metric to measure acoustic models is
frame accuracy, which is measured on each $s_t$ prediction for all
$t$. Generally, this metric correlates with the actual metric of
interest, the Word Error Rate (WER). However, since computing WER
involves using a language model and tuning the decoding parameters for
every change in the acoustic model, we decided to report frame
accuracy. Table~\ref{tab:speech} clearly shows the positive effect of
dropout by improving frame accuracy. Not surprisingly, the training
frame accuracy drops due to the noise added during training, but as is
often the case with dropout, this yields models that generalize better
to unseen data. Note that the testing set is easier than the training
set, given its higher accuracy.  We report the performance of an LSTM
on an internal Google Icelandic Speech dataset, which is relatively
small, so overfitting is a greater concern. 

\begin{table}[t]
  \small
  \centering
  \renewcommand{\arraystretch}{1.15}
  \begin{tabular}{lll}
    \hline
     Model & Training set & Validation set \\
    \hline
    Non-regularized LSTM & 71.6 & 68.9 \\
    Regularized LSTM & 69.4 & {\bf 70.5} \\
    \hline
  \end{tabular}
  \caption{Frame-level accuracy on Icelandic Speech Dataset. {\bf size of dataset?}}
  \label{tab:speech}
\end{table}


\subsection{Machine translation}
\label{sec:trans}

We report the performance of the LSTM on a machine translation task.
We formulate the translation task as a language modelling task, where
the LSTM is trained to assign high probability to the correct
translation given a source sentence.  Thus, the LSTM is trained on
sequences of the form \texttt{(source sentence, target sentence)}
\cite{mt_paper,cho2014learning}. We compute the translations using a simple
left-to-right decoder; see \cite{mt_paper} for more details.  We ran
this experiment on the WMT'14 English to French dataset, on the
``selected'' subset from \cite{wmt_joint} which has 340M French words
and 304M English words.  Our LSTM that has 4 hidden layers, where both
the layers and the word embeddings are 1000-dimensional, and where the
English vocabulary has 160,000 words and the French vocabulary has
80,000 words, as in \cite{mt_paper}. We found the optimal dropout
probability to be 0.2.  Table \ref{tab:mt} shows the comparative
performance of both LSTMs.  While this particular LSTM does not beat
the standard SMT system \cite{lium}, this result clearly shows that
dropout greatly improves the performance of the LSTM.  

\begin{table}[t]
  \small
  \centering
  \renewcommand{\arraystretch}{1.15}
  \begin{tabular}{lll}
    \hline
     Model & Test perplexity & Test BELU score \\
    \hline
    Non-regularized LSTM & 5.8 & 25.9 \\
    Regularized LSTM & 5.0 &  29.03 \\
    \hline
    LIUM system &  &  33.30 \\
    \hline
  \end{tabular}
  \caption{Results on the English to French translation task. }
  \label{tab:mt}
\end{table}



\section{Discussion}

We presented a simple application of dropout to LSTMs that resulted in
large performance boosts on many separate domains.  Our results make dropout
useful for RNNs, and our results suggest that this type of dropout could improve
performance on a wide variety of applications.



%% We present performance boosts with methods, which we 
%% only intuitively understand. We think, that it crucial
%% to derive our models analytically, rather than based only on
%% intuition. However, so far this problems seem to be very difficult to
%% tackle.


\bibliography{bibliography}
\bibliographystyle{icml2014}

\end{document} 

